\documentclass[a4paper,12pt]{article}

% For the russian see 
% http://tex.stackexchange.com/questions/816/cyrillic-in-latex
% http://tex.stackexchange.com/a/72690/4992


\usepackage[T2A,T1]{fontenc}
\usepackage[utf8]{inputenc}
\usepackage[russian,english]{babel}

\usepackage{graphicx}
\usepackage{amsmath}
\usepackage{hyperref}
\hypersetup{
    colorlinks=false,
    pdfborder={0 0 0},
}

\usepackage[backend=bibtex]{biblatex}
\bibliography{bibliography.bib}



\title{A bibliography for the problem of non-terminating game of Beggar-My-Neighbour}
\author{Alessandro Gentilini\thanks{alessandro.gentilini@gmail.com}}


\begin{document}
\maketitle

\begin{abstract}
A bibliography for the problem of non-terminating game of Beggar-My-Neighbour.
\end{abstract} 

\section{A bibliography for the problem of non-terminating game of 
Beggar-My-Neighbour}

In \cite[p.164]{paulhus1999beggar}:
\begin{quotation}
If $C$ is a full deck of cards, does ${D_{2}}^{'}(C)$ have a cycle? We leave 
this question unanswered except to say that we have been unable to find one in 
3.2 billion randomly chosen deals.
\end{quotation}

% http://library.msri.org/books/Book42/files/guy.pdf
In \cite[p.472]{nowakowski2002more}:
\begin{quotation}
56. Are there any draws in \textbf{Beggar-my-Neighbor?}\\ 

[Two players deal single cards in turn onto a common stack. If a court card (J,
Q, K, A) is dealt, the next player must cover it with respectively 1, 2, 3, 4 
cards. If one of these is a court card, the obligation to cover reverts to the 
previous player. If they are not court cards, the previous player acquires the 
stack, which he inverts and places beneath his own hand, and starts dealing 
again. A player loses if she is unable to play.]\\

This problem reappears periodically. It was one of Conway’s 
‘anti-Hilbert problems’ about 40 years ago, but must have suggested itself to 
players of the game over the several centuries of its existence.\\

Marc Paulhus [1999] exhibited some cycles with small decks, and used a computer 
to show that there were no cycles when the game is played with a half-deck, 
although the addition or subtraction of two non-court cards produced cycles. 
Michael Kleber found an arrangement of two 26-card hands which required the 
dealing of 5790 cards before a winner was declared.
\end{quotation}

In \cite[p.892]{winning_ways_2004}:
\begin{quotation}
Strip-Jack-Naked, or Beggar-My-Neighbour **1\\

Another problem that took almost 47 years to solve concerns this old 
children’s game. Each of the two players starts with about half of the cards 
(held face-down), which they alternately turn over onto a face-upwards “stack” 
on the table, until one of them (who's now “the commander”) first deals one of 
the “commanding cards” (Jack, Queen, King, or Ace).

After one of these has been dealt, the other player (now “the responder”) turns 
over cards continuously until EITHER. **2 a new commanding card appears (when 
the players change roles **3) or respectively 1, 2, 3, or 4 non-commanding cards 
have been turned over. In the latter case, the commander turns over the stack 
and ajoins it to the bottom of his hand. The responder then starts the formation 
of a new stack by turning over his next card, and play continues as before.

A player who acquires all the cards is the winner and in real games, it seems 
that someone always does win. The interesting mathematical question, posed by 
one of us many years ago, was “is it really true that the game always ends?” 
Marc Paulhus has recently found the answer to be “no!”. About 1 in 150,000 games 
(played with the usual 52 cards) goes on forever.

We are fairly confident that no one person has played the game anything like 
that number of times, so the chance (with random shuffling) of experiencing a 
non-terminating game in a lifetime’s play must be very small indeed.

Just as surely, however, the total number of times this game has been played by 
the World’s **4 children must be significantly larger than 150,000, so many of 
them will have been theoretically non-terminating ones. We imagine, though, that 
in practice most of them actually did terminate because someone made a mistake.
\end{quotation}

% http://library.msri.org/books/Book56/files/61guy.pdf
In \cite[p.483]{albert2009games}:%"Unsolved problems in Combinatorial Games" 465-489
\begin{quotation}
\textbf{D7(56).} Are there any draws in Beggar-my-Neighbor? Marc Paulhus
showed that there are no cycles when using a half-deck of two suits, but the 
problem for the whole deck (one of Conway’s “anti-Hilbert” problems) is still 
open.
\end{quotation}

% http://mathcs.pugetsound.edu/~mspivey/War.pdf
In \cite{spivey2010cycles}:
\begin{quotation}
Berlekamp, Conway, and Guy also report in Vol. 4 of Winning Ways for Your 
Mathematical Plays [2, p. 892] that Marc Paulhus has shown that the similar game 
of Beggar-My-Neighbor can cycle, although the cycles are rare: About 1 in 
150,000 games played with the usual 52-card deck cycle. (For more on 
Beggar-My-Neighbor, see Paulhus [5].)
\end{quotation}

In the abstract of \cite{russians_arxiv}:
\begin{quotation}
It is proved that in card games similar to 'Beggar-my-neighbour' the 
mathematical expectation of the playing time is finite, provided that the player 
who starts the round is determined randomly and the deck is shuffled when the 
trick is added. The result holds for the generic setting of the game.
\end{quotation}

In the abstract of \cite{russians_2013}:
\begin{quotation}
For card games of the Beggar-My-Neighbor type, we prove finiteness of the 
mathematical expectation of the game duration under the conditions that a player 
to play the first card is chosen randomly and that cards in a pile are shuffled 
before being placed to the deck. The result is also valid for general-type 
modifications of the game rules. In other words, we show that the graph of the 
Markov chain for the Beggar-My-Neighbor game is absorbing; i.e., from any vertex 
there is at least one path leading to the end of the game.
\end{quotation}

\printbibliography
\end{document}
